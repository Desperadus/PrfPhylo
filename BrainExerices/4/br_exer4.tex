\documentclass{article}
\usepackage{geometry}
\geometry{a4paper, margin=1in}
\usepackage{amsmath}
\usepackage{graphicx}
\usepackage{hyperref}

\title{Brain Excercise 4}
\author{Tomáš Jelínek}
\date{\today}

\begin{document}

\maketitle

\begin{figure}[h]
    \centering
    \includegraphics[width=\textwidth]{Brain_exercise_4.pdf}
\end{figure}

\section*{Odpověd}

Myslím, že by to mohlo vzniknout tak, že výskyt SINE v daném lokusu pro společného předka byl polymorfnismus. U gorily došlo k vyštěpení z předků, kteří jej primárně měli a nebo/+pak se případně driftem SINE plně chytnul - to samé pak později u člověka.
Ovšem u šimpanzů se stal prakticky opak - ve vyštěpující se populaci SINE nebyl a tak driftem vymizel. 

\end{document}