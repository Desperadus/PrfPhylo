\documentclass{article}
\usepackage{geometry}
\geometry{a4paper, margin=1in}
\usepackage{amsmath}
\usepackage{graphicx}
\usepackage{hyperref}
\usepackage{booktabs}
% Make czech characters work
\usepackage[czech]{babel}
%add package for python code
\usepackage{listings}
\usepackage{color}
\usepackage{float}
\usepackage{subcaption}
\usepackage{tikz}
\usepackage{pgfplots}
\usepackage{pgfplotstable}
\usepackage{filecontents}

\pgfplotsset{compat=1.18}

\lstset{
    language=Python,
    basicstyle=\ttfamily\small,
    keywordstyle=\color{blue},
    stringstyle=\color{red},
    commentstyle=\color{green},
    showstringspaces=false,
    numbers=left,
    numberstyle=\tiny\color{gray},
    breaklines=true,
    frame=single,
    inputencoding=utf8
}

\title{Brain excercise 7}
\author{Tomáš Jelínek}
\date{\today}

\begin{document}

\maketitle

\section*{Odpověď}

\subsection*{Zakořeněné}

Počet zakořeněných topologií pro \(n\) taxonů je dán vzorcem:
\[
R(n) = (2n - 3)!! = (2n-3) \cdot (2n-5) \cdot (2n-7) \cdots 1
\]

\textbf{Odvození:} 
Pro počet zakořeněných stromů \(R(n)\) použiji rekurentní přístup - já tohle četl někde už v prváku, takže nejde o můj důkaz.

1. \textbf{Základní případ:} Pro \(n = 2\) máme přesně jednu topologii (jeden strom se dvěma taxony spojenými hranou), takže:
   \[
   R(2) = 1
   \]

2. \textbf{Rekurentní krok:} Při přidání \(n\)-tého taxonu ke stromu s \((n-1)\) taxony lze tenhle taxon k jakykoliv hraně. Strom se \((n-1)\) taxony má \(2(n-1) - 1 = 2n - 3\) hran.

   Proto počet možností roste podle:
   \[
   R(n) = (2n - 3) \cdot R(n-1)
   \]

3. \textbf{Vyřešení rekurence:} Je prakticky hned vidět...
   \[
   R(3) = (2 \cdot 3 - 3) \cdot R(2) = 3 \cdot 1 = 3
   \]
   \[
   R(4) = (2 \cdot 4 - 3) \cdot R(3) = 5 \cdot 3 = 15
   \]
   
   Obecně tedy platí:
   \[
   R(n) = (2n - 3)!! 
   \]

\subsection*{Nezakořeněné}
Počet nezakořeněných topologií pro \(n\) taxonů je:
\[
U(n) = (2n - 5)!!
\]

\textbf{Odvození:} 
Pro nezakořeněné binární stromy s \(n\) taxony je odvození podobné jako u zakořeněných stromů. V nezakořeněném stromu s \(n\) taxony je počet hran \(2n - 5\), což je o dvě méně než počet hran ve zakořeněném stromu. To je proto, že odstraněním kořene z zakoř. stromu snížíme počet hran o dvě (jednu pro kořenovou hranu a jednu pro první vnitřní hranu).


Proto platí:
\[
U(n) = (2n - 5)!! 
\]

\end{document}