\documentclass{article}
\usepackage{geometry}
\geometry{a4paper, margin=1in}
\usepackage{amsmath}
\usepackage{graphicx}
\usepackage{hyperref}
% Make czech characters work
\usepackage[czech]{babel}
\usepackage{bm}

\title{Brain excercise 5}
\author{Tomáš Jelínek}
\date{\today}

\begin{document}

\maketitle

\section*{Odpověď}

Víme, že po nekonečně dlouhé době mají ty sekvence \textbf{35\% C, 35\% G, 15\% A a 15\% T}.
\\
V první sekvecni uvidímě na 35\% pozicích G a to, že v druhé sekvecni k nějakému danému G bude G je tedy opět 35\%, to samé pro C a pro T s A to bude 15\%.
\\
\\
Tedy počet schodných pozic bude $0.35 \cdot 0.35 \cdot 2 + 0.15 \cdot 0.15 \cdot 2 = 0.29$.
\\
\\
No a tím padém bude \boldmath{$p=1-0.29 = 0.71$}.
\end{document}
