\documentclass{article}
\usepackage{geometry}
\geometry{a4paper, margin=1in}
\usepackage{amsmath}
\usepackage{graphicx}
\usepackage{hyperref}
\usepackage{booktabs}
% Make czech characters work
\usepackage[czech]{babel}
%add package for python code
\usepackage{listings}
\usepackage{color}
\usepackage{float}
\usepackage{subcaption}
\usepackage{tikz}
\usepackage{pgfplots}
\usepackage{pgfplotstable}
\usepackage{filecontents}

\pgfplotsset{compat=1.18}

\lstset{
    language=Python,
    basicstyle=\ttfamily\small,
    keywordstyle=\color{blue},
    stringstyle=\color{red},
    commentstyle=\color{green},
    showstringspaces=false,
    numbers=left,
    numberstyle=\tiny\color{gray},
    breaklines=true,
    frame=single,
    inputencoding=utf8
}

\title{Compulsory Assignment 6}
\author{Tomáš Jelínek}
\date{\today}

\begin{document}

\maketitle
\section*{Odpověď}

Vlastně jsem jen do vzorce z prezentace dosadil ty naše čísla z úkolu:
\[
\begin{aligned}
&\text{P}_o(t) = \frac{3}{10} \left( \frac{1}{4} + \frac{3}{4} e^{-0t} \right) + \frac{7}{10} \left( \frac{1}{4} + \frac{3}{4} e^{-\frac{10t}{7}} \right) \\
&\text{P}_o(t) = \frac{3}{10} + \frac{7}{10} \left( \frac{1}{4} + \frac{3}{4} e^{-\frac{10t}{7}} \right) = \frac{21}{40} e^{-(10t)/7} + \frac{19}{40}
\end{aligned}
\]

Ve finále, že zůstane nezměněný budou ty 3/10 (šance, že se nachází v invariantní části) + to, že se budou stejný v měnicí se části - klasicky dle Jukes Cantorova modelu * 7/10 (šance, že je v tý měnící se části).

\end{document}
