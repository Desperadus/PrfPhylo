\documentclass{article}
\usepackage{geometry}
\geometry{a4paper, margin=1in}
\usepackage{amsmath}
\usepackage{graphicx}
\usepackage{hyperref}
\usepackage{booktabs}
% Make czech characters work
\usepackage[czech]{babel}
%add package for python code
\usepackage{listings}
\usepackage{color}
\usepackage{float}
\usepackage{subcaption}
\usepackage{tikz}
\usepackage{pgfplots}
\usepackage{pgfplotstable}
\usepackage{filecontents}

\pgfplotsset{compat=1.18}

\lstset{
    language=Python,
    basicstyle=\ttfamily\small,
    keywordstyle=\color{blue},
    stringstyle=\color{red},
    commentstyle=\color{green},
    showstringspaces=false,
    numbers=left,
    numberstyle=\tiny\color{gray},
    breaklines=true,
    frame=single,
    inputencoding=utf8
}

\title{Compulsory Assignment + Brain excercise 11}
\author{Tomáš Jelínek}
\date{\today}

\begin{document}

\maketitle
\section*{Odpověď - Compulsory}

Pouze otec číslo 2 je potencílním rodičem, ostatní lze vyloučit, protože dítě má alely, co oni ne a nejsou od matky.

\section*{Odpověď - Brain excercise}

Vidíme jaké alely jsou od matky a jaké od otce (ta co prostě není od matky). A lze zkonstruovat tabulku frekvencí pro dané alely z toho webu:

\begin{table}[h!]
    \centering
    \caption{Frekvence alel od otce pro Česko and Evropu}
    \label{tab:allele_frequencies}
    \begin{tabular}{|l|c|c|}
    \hline
    \textbf{Alela}        & \textbf{Česko}   & \textbf{Evropa}  \\ \hline
    D1S1656 - 12          & \( 1.6500 \times 10^{-1} \) & \( 1.2778 \times 10^{-1} \) \\ \hline
    D2S1338 - 19          & \( 1.0750 \times 10^{-1} \) & \( 1.0874 \times 10^{-1} \) \\ \hline
    D2S441 - 9            & \( 2.5000 \times 10^{-3} \) & \( 1.3194 \times 10^{-3} \) \\ \hline
    D3S1358 - 19          & \( 7.5000 \times 10^{-3} \) & \( 1.2017 \times 10^{-2} \) \\ \hline
    D8S1179 - 16          & \( 2.0000 \times 10^{-2} \) & \( 2.5490 \times 10^{-2} \) \\ \hline
    \textbf{Produkt}      & \( 6.6515625 \times 10^{-9} \) & \( 5.61558 \times 10^{-9} \) \\ \hline
    \end{tabular}
\end{table}

V tabulce vidíme ony frekvence lidí s danými alelami - tedy potenciálními otci děleno dvěmi.
\\
\\
\subsection*{Počty lidí v Česku a Evropě} 

\subsubsection*{Česko}
Počet čechů jenž jsou potenciálními otci bude tedy: 

\[
10800000/2 \times 6.6515625 \times 10^{-9} = 0.036...
\]
(Počítám že v česku je cca 10.8 milionu lidí a polovina jsou muži)

\subsubsection*{Evropa}
Počet evropanů jenž jsou potenciálními otci bude tedy:

\[
741000000/2 \times 5.61558 \times 10^{-9} = 2.08...
\]
(Počítám že v evropě je cca 741 milionů lidí a polovina jsou muži)

\end{document}
