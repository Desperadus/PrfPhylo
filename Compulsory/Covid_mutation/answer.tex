\documentclass{article}
\usepackage{geometry}
\geometry{a4paper, margin=1in}
\usepackage{amsmath}
\usepackage{graphicx}
\usepackage{hyperref}
% Make czech characters work
\usepackage[czech]{babel}

\title{Compulsory Assignment 5}
\author{Tomáš Jelínek}
\date{\today}

\begin{document}

\maketitle

\section*{Odpověď}

Mutace mají Poissonovu distribuci - pro kterou je potřeba mít parametr $\lambda$.
Ze zadaní víme, že mutační rychlost covidu jsou dvě mutace za měsíc, což je 24 mutací za rok.
Tedy $\lambda = 24$.

Poissonova distribuce je dána vztahem:

\begin{equation}
    P(X=k) = \frac{e^{-\lambda} \lambda^k}{k!}
\end{equation}

kde $k$ je počet mutací co pozorujeme.
\\
\\
Nyní stačí do vzorce dosadit počet let a počet mutací, které chceme pozorovat a získame ony kýžené pravděpodobnosti.

\subsection*{Pravděpodobnost 100 mutací za rok u Covidu}

\begin{equation}
    P(X=100) = \frac{e^{-24} 24^{100}}{100!} \approx 4.25 \times 10^{-31}
\end{equation}

\subsection*{Pravděpodobnost jedné mutace za rok u Covidu}

\begin{equation}
    P(X=1) = \frac{e^{-24} 24^{1}}{1!} \approx 9.06 \times 10^{-10}
\end{equation}

\subsection*{Ostatní viry}

Pro ostaní viry bychom postupovali obdobně, jen bychom měli jiný parametr $\lambda$ a to pro Influenzu $\lambda = 48$ a pro HIV $\lambda = 96$.
\\
\\
(takže to by bylo $P(X=k) = \frac{e^{-48} 48^k}{k!}$ pro Influenzu a $P(X=k) = \frac{e^{-96} 96^k}{k!}$ pro HIV)

\end{document}
