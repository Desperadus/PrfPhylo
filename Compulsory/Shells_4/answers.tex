\documentclass{article}
\usepackage{geometry}
\geometry{a4paper, margin=1in}
\usepackage{amsmath}
\usepackage{graphicx}
\usepackage{hyperref}

\title{Compulsory Assignment 4}
\author{Tomáš Jelínek}
\date{\today}

\begin{document}

\maketitle

\section*{Otázky a Odpovědi}

\begin{enumerate}
    \item \textbf{What protects the proteins in question from degradation, so that they are detectable after thousands of years?}

    Proteiny jsou uzavřeny uvnitř minerálních krystalů, což vytváří takzvaný "\textit{closed system}" a ty proteiny se pak dle článku nazývají "\textit{intracrystalline proteins}". Ta minerální schárnka je chraní tedy před vnější kontaminací a před možnou degradací (například mikriobiální).

    \item \textbf{What is the weight fraction of proteins in the shell mass?}

    Organická složka tvoří něco mezi $0.01\%$ až $2\%$ hmotnosti. Ale ta \textit{intracrystalline proteinová} část je pouze $0.001\%$ až $0.01\%$ hmotnosti schránky.
    Pro \textit{cross-lamellární} schránky je obsah org. materiálu, ale klidně pouze cca $0.004\%$ hmotnosti.

    \item \textbf{Which method of sample preparation listed in table 1 was considered as best?}

    Metoda číslo $12$: \textit{Ic\_EDTA\_SP3}

    \item \textbf{How many times each sample was measured on mass spectrometry?}

    2x a spektra byla následně zpruměrována pro každý vzorek.

    \item \textbf{List some common protein contaminants in the samples.}

    Trypsin, keratin, a $\alpha$-cyano MALDI matrix.

    \item \textbf{How many species-specific peaks were detected in the analysed shells?}

    Záleželo na druhu, z tabulky 2 lze vyčíst, že pro jednotlivé druhy jsou počty:
    \begin{table}[h!]
        \centering
        \begin{tabular}{|l|c|}
        \hline
        \textit{Druh} & \textit{Počet specifických peaků} \\ \hline
        \textit{Ostrea edulis} & 15 \\ \hline
        \textit{Pecten maximus} & 6 \\ \hline
        \textit{Spondylus gaederopus} & 13 \\ \hline
        \textit{Patella vulgata} & 24 \\ \hline
        \textit{Phorcus turbinatus} & 18 \\ \hline
        \textit{Unio pictorum} & 10 \\ \hline
        \textit{Pseudunio auricularius} & 14 \\ \hline
        \end{tabular}
    \end{table}
    \\
    \\
    Celkem tedy 100 markerů.

\end{enumerate}



\end{document}