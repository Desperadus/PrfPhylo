\documentclass{article}
\usepackage{geometry}
\geometry{a4paper, margin=1in}
\usepackage{amsmath}
\usepackage{graphicx}
\usepackage{hyperref}
\usepackage{booktabs}
% Make czech characters work
\usepackage[czech]{babel}
%add package for python code
\usepackage{listings}
\usepackage{color}
\usepackage{float}
\usepackage{subcaption}
\usepackage{tikz}
\usepackage{pgfplots}
\usepackage{pgfplotstable}
\usepackage{filecontents}

\pgfplotsset{compat=1.18}

\lstset{
    language=Python,
    basicstyle=\ttfamily\small,
    keywordstyle=\color{blue},
    stringstyle=\color{red},
    commentstyle=\color{green},
    showstringspaces=false,
    numbers=left,
    numberstyle=\tiny\color{gray},
    breaklines=true,
    frame=single,
    inputencoding=utf8
}

\title{Compulsory Assignment 10}
\author{Tomáš Jelínek}
\date{\today}

\begin{document}

\maketitle
\section*{Odpověď}

Zde je vyjádření časů:

\begin{align*}
    v_1 &= v_2 \\
    v_4 &= v_5 \\
    v_1 + v_3 &= v_4 + v_6 \\
    \hline
    v_7 &= v_8 \\
    v_{10} + v_7 &= v_9 \\
    v_9 + v_{11} &= v_{12} + v_3 + v_1 \\
    \end{align*}

z toho prohlásím že: $v_1, v_3, v_4, v_7, v_{10}, v_{12}$ jsou nezávislé, tedy budu mít 6 stupňů volnosti.

Na základě mého kódu vycházi p-value $0.066582206320095829$ což je více než $0.05$ a tedy nezamítáme nulovou hypotézu.

\subsection*{Kód}
% show code
\lstinputlisting{lrt.py}

\end{document}
