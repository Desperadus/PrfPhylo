\documentclass{article}
\usepackage{geometry}
\geometry{a4paper, margin=1in}
\usepackage{amsmath}
\usepackage{graphicx}
\usepackage{hyperref}
\usepackage{booktabs}
% Make czech characters work
\usepackage[czech]{babel}
%add package for python code
\usepackage{listings}
\usepackage{color}
\usepackage{float}
\usepackage{subcaption}
\usepackage{tikz}
\usepackage{pgfplots}
\usepackage{pgfplotstable}
\usepackage{filecontents}

\pgfplotsset{compat=1.18}

\lstset{
    language=Python,
    basicstyle=\ttfamily\small,
    keywordstyle=\color{blue},
    stringstyle=\color{red},
    commentstyle=\color{green},
    showstringspaces=false,
    numbers=left,
    numberstyle=\tiny\color{gray},
    breaklines=true,
    frame=single,
    inputencoding=utf8
}

\title{Compulsory Assignment 6}
\author{Tomáš Jelínek}
\date{\today}

\begin{document}

\maketitle

\section*{Odpověď}

Mějme tedy onu matici vzdáleností:
\\
\\
\begin{tabular}{lrrrr}
    \toprule
     & A & B & C & D \\
    \midrule
    A & 0.000000 & 0.800000 & 0.900000 & 0.600000 \\
    B & 0.800000 & 0.000000 & 0.500000 & 0.400000 \\
    C & 0.900000 & 0.500000 & 0.000000 & 0.500000 \\
    D & 0.600000 & 0.400000 & 0.500000 & 0.000000 \\
    \bottomrule
\end{tabular}
\\
\\
A uvedené evoluční stromy jsem převedl do newick formátu:
\begin{enumerate}
    \item (((B:0.2, C:0.3):0.1, D:0.1):0.1, A:0.4);
    \item (((B:0.2, D:0.2):0.05, C:0.25):0.13, A:0.383);
\end{enumerate}

Pak jsem si to napsal v pythonu ty vypočty (ručně by to bylo určitě rychlejší, ale tak je to trenink):

Výsledky:
\begin{enumerate}
    \item \textbf{Minimum evolution:}
    \begin{itemize}
        \item Q1 = 1.2
        \item Q2 = 1.213
    \end{itemize}
    \item \textbf{Least Squares:}
    \begin{itemize}
        \item Q1 = 0
        \item Q2 = 0.046707
    \end{itemize}
\end{enumerate}

Jak na základě minimální evoluce, tak na základě metody nejmenších čtverců, je vidět, že preferovaný strom je ten první.
\\
\\
Zde dole máte kód jak jsem to počítal

\subsection*{Python kód}

\lstinputlisting[language=Python, inputencoding=utf8]{answer.py}

\end{document}
